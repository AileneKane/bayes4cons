\documentclass[11pt]{article}
\usepackage[top=1.00in, bottom=1.0in, left=1in, right=1in]{geometry}
\renewcommand{\baselinestretch}{1.1}
\usepackage{graphicx}
\usepackage{natbib}
\usepackage{amsmath}
\usepackage{parskip}

\def\labelitemi{--}
\parindent=0pt

\begin{document}
\bibliographystyle{/Users/Lizzie/Documents/EndnoteRelated/Bibtex/styles/besjournals}
\renewcommand{\refname}{\CHead{}}

% \setlength{\parindent}{0cm}
% \setlength{\parskip}{5pt}

\section*{Overview}

We propose follow simple example of Wade 2000 (with citation to it) with a climate change twise... 
\begin{itemize}
\item Back story for our example ... 
\begin{itemize}
\item Imagine you have sampled 10 populations of a species across its latitudinal range
\item Populations are increasing at the leading edge, but declining fast at the trailing edge.
\item However, there's different sample sizes (years of data) at each site and there happens to be much less at the most trailing edge population
\end{itemize}
\item What we'll do. ... 
\begin{itemize}
\item We'll simulate from a basic logistic growth model and add equal noise across all populations, then reduce the sample size ($n$ of years) variably but the most for the most trailing and make sure one a little up of that has LOTS of data
\item We'll show trends lines with error and asterisks for the NHT look and each one will be paired with a posterior with 0 highlighted by a dashed line ({\bf we could also show the true slope?} Though I am not sure how to calculate that ... given the underlying model is non-linear, but we could estimate on the data without the noise added). We will make sure the $x$ axis is the same across all the posteriors so the decline is apparent. 
\end{itemize}
\item What this will show ... 
\begin{itemize}
\item The most declining population will not have a significant slope so will not be flagged for concern under NHT and p-values but will look concerning with the posterior.
\item Instead the NHT will make the well sampled but barely declining population look most concerning. 
\end{itemize}
\end{itemize}


\section*{Starter text ...}

In some cases, Bayesian approaches can lead to different conclusions than Fisherian approaches, such as null hypothesis testing (Wade 2000) ...

\end{document}

